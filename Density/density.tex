.AUTOPARAGRAPH          ! Blank in space 1 implies new paragraph
.SET PARAGRAPH 5        ! Indent 5 spaces
.JUSTIFY                ! Ragged right margin
.FLAG BOLD
.CC
.LEFT MARGIN 10                 ! One inch
.RIGHT MARGIN 72        ! One inch - do not reset greater than 78!
.PAGE SIZE 78,78        ! 78 lines X 78 spaces, Max 81 X 79 for 8 1/2 X 11
.LEFT MARGIN 3                 ! One inch
.RIGHT MARGIN 82        ! One inch - do not reset greater than 78!
.PAGE SIZE 60,72        ! 78 lines X 78 spaces, Max 81 X 79 for 8 1/2 X 11
.TAB STOPS 14,22,30,38,46,54,62,78 ! Default values, will print exactly as
.lo 1,1
.xl
.hd
.C
PREFACE
 DENSITY is the second program in a series of semi-empirical
programs designed to allow MNDO and MINDO/3 calculations to be
carried out, the first member of the series being  MOPAC. 
 MOPAC is a large general-purpose program for the study of
the electronic structures of molecules. The user is expected to
be familiar with the use of MOPAC before starting to work with
DENSITY.
In order to use DENSITY properly it is essential to have a
copy of MOPAC, version 3.0. If the user has a version higher
than 2.0, the upgrade to allow generation of files suitable 
for DENSITY can be made. (Files to check that the update
has been done successfully are provided)
 While MOPAC jobs use large amounts of computer time, DENSITY
calculations typically only last one or two minutes; therefore several
plots can readily be run off.
 Graphical output is terminal-dependent, so users may encounter
difficulty there. A general utility program for use with MOPAC,
DRAW is capable of accepting DENSITY output and drawing
.x DRAW
plots on Tektronix equipment. Ideally, a user would have access to
all three programs: MOPAC as the main tool, DRAW as a
general utility for manipulating MOPAC files and generating
data for a graphics terminal, and DENSITY
for generating graphical output.
 
 

.ch  INTRODUCTION
.LEFT MARGIN 10                 ! One inch
 The program DENSITY is designed to generate maps of electron density distribution
and molecular orbital intensity within molecules. Two main modes of operation
.x Stand alone
are provided: manual data input, which is a "stand-alone" mode, and normal
.x Normal mode
input, which assumes the existence of large unformatted data-files produced 
.x MOPAC, files made by
.x Files, made by MOPAC
by, e.g. MOPAC.
 To generate the large unformatted data-file, MOPAC should be run using
the key-word GRAPH. At present, MOPAC can only produce graphics files
.x RHF, limitation to
in the RHF mode, the UHF capability having not yet being written.
 For small molecules, e.g., up to ethylene, DENSITY could be run on-line,
.x Running, on line
.x On line running
but for larger systems, or for multiple plots, batch-mode operation is 
.x Running, batch mode 
.x Batch mode running
recommended. 
  Although DENSITY has a very simple data-input, users are warned that
care is needed in precisely wording a request. As an example, in learning
how to use DENSITY a user might want to plot the highest-filled pi orbital
.x Pi systems
in benzene, and choose the plane to be that of the carbon atoms. At first
sight this appears reasonable, but when one recalls that the pi system has
a node in this plane, another choice of plane is seen to be essential.
A better choice of plane would be one parallel to the plane of the carbon 
atoms, but 0.5 to 1.0 Angstroms above it.
 Similarly, if only a detail of the molecule is to be studied, it is
wasteful to plot the whole molecule. The picture is essentially made
.x Pixel density
up of 2,500 pixels, so fine detail can very easily be lost if the whole
molecule is used.
 With one important exception the whole program is written using the
FORTRAN-77 standard, so translation to allow DENSITY to run
on different machines should not be difficult. FORTRAN-77 does not
support graphical functions, and an interface, designed to allow
users to easily write their own interfaces, was written.
.ch DATA INPUT
.LEFT MARGIN 10                 ! One inch
 Rather than fully define the input, a worked example will be given,
and other data-files can be generated by analogy.
.hl Example of Data
 Pi system in ethylene. Two pictures are to be drawn. 
The first plot is in the plane parallel 
to the plane of the molecule, the second one is in the plane perpendicular to
the C-C axis.(A MOPAC file is assumed to exist)
.lt
               Example
Line 1 :    
Line 2 :  Ethylene pi system
Line 3a:  CENTER=(0.7,0.0,0.5) LINE=(0.0,0.0,1.0) EDGE=2.0 HOMO
Line 3b:  CENTER=(0.7,0.0,0.5) LINE=2 EDGE=2.0 HOMO
Line 4 :

Notes:
.el
Line 1 : key-words controlling the mode of operation.
.S 1
Line 2 : Title of plot. Up to 79 characters can be used.
.S 1
Line 3a: Key-words. A full list of key-words, and their definitions, is 
given in the next Chapter.
.S 1
Line 3b: Key-words for the second picture. Multiple pictures are not yet 
de-bugged fully - there is not much demand for them.
.S 1
Line 4 : End the input with a blank line.
.ch KEY-WORDS
.LEFT MARGIN 10                 ! One inch
 Two sets of key-words are provided. At the start of a run the user must
provide information about the source of the molecular data. At present,
this data can come from one of two sources. These are (a) off the data-file
itself, which is the MANUAL mode, and (b) from a disk or restart-type file,
.x MANUAL
which was created by an earlier run using MOPAC.
.s 2
.C
 FIRST SET OF KEY-WORDS

.LT
 DEBUG       -  Print part of working of DENSITY.
 DMAT        -  Lower half triangle of density matrix to be read in
                (Used with MANUAL only)
 MANUAL      -  Stand-alone  mode. All data supplied from datafile.
 M.O.        -  A molecular orbital to be used as source of density
                matrix. (Used with MANUAL only)
 TIGER       -  A pixel map, in bytes, suitable for the Paper Tiger
                to be generated. (Default is no Paper Tiger picture)
.el

.tp16
.hl Definitions of First set of Key-Words
.s 2
.tp 14
.c
DEBUG
.x DEBUG, Definition of
 The density matrix or molecular orbital used by the density calculation
will be printed if DEBUG is specified.
.tp 15
.c
 MANUAL
.x MANUAL, Definition of
 This word was used in the writing and de-bugging of DENSITY, 
but now is only intended for testing and demonstration purposes. 
  When MANUAL is specified, either DMAT or M.O.
.x M.O.
.x DMAT
must be specified on the same line as MANUAL.
 If MANUAL is specified, the following data must be supplied on the
next few lines:
.ls
.tp 12
.le;Number of atoms, orbitals, and electrons (NATOMS, NORBS, NELECS). 
One line, free format, integer.
.tp 12
.le;Cartesian coordinates for each atom, one atom per line, free format,
real, in order x, y, z. There will be NATOMS of these.
.tp 12
.le;Labels for all atoms. For example, all carbon atoms might be called
"1", all hydrogen atoms "2", etc. Free format, integer. There will be NATOMS
of these numbers. The numbers must be 1 to NUNI, NUNI being the number
of unique atom-types in the molecule. 
.tp 12
.c
 Examples
.lt
   Compound         Atom Labels        NUNI Line would read
  Methane         C: 1,   H: 2          2   1 2 2 2 2         CH4
  Formic Acid     H: 1,   C: 2,  O: 3   3   1 2 3 3 1         HCOOH
  Thiomethanol    C: 1,   H: 2,  S: 3   3   1 2 2 2 3 2       CH3SH
.el
.tp 12
.le;Atomic Numbers for all unique atoms. Free format, integer. There are
NUNI of these numbers. If atom of type 1 is a carbon atom, then the 
corresponding atomic number would be 6. Using the examples above, we have:
.lt
                  Atomic Numbers line would read
     Compound   Unique atom:  1  2  3
     Methane                  6  1
     Formic Acid              1  6  8
     Thiomethanol             6  1 16
.el
.tp 12
.le;Principal Quantum numbers. Free format, integer. There are
NUNI of these numbers.
.c
 Examples
.lt
               Principal Quantum Numbers line would read
               
     Compound   Unique atom:  1  2  3
     Methane                  2  1
     Formic Acid              1  2  2
     Thiomethanol             1  2  3
.el
.tp 12
.le;Number of orbitals for all unique atoms. Free format, integer, there are
NUNI of these numbers.
.c
 Examples
.lt
                Number of Orbitals line would read
                
     Compound   Unique atom:  1  2  3
     Methane                  4  1
     Formic Acid              1  4  4
     Thiomethanol             1  4  9
.el
.tp 12
.le;Orbital exponents for all unique atoms, format is 3F10, three numbers
per atom (one each for s, p, and d orbital exponents); if p or d is absent, these
can be left blank.
.c
 Examples
.lt
    Compound         Unique atom       Exponents line would read
     Methane              1        1.78       1.78
                          2        1.30
     Thiomethanol         1        1.78       1.78
                          2        1.3
                          3        2.4        2.1        1.0
.el
.tp 12
.le;If M.O.  is specified, the eigenvector coefficients are read in
at this point. Free format, there
are NORBS of these. Otherwise, if DMAT specified, the lower-half
.x DMAT
of the density matrix, in free-format, is read in. There would be 
(NORBS_*(NORBS+1))/2 of these numbers.
.els
.tp 22
.c
Full Example of Manual Data for Formic Acid
.lt
Line  1:      MANUAL       M.O.                   (Key-words
Line  2:      FORMIC ACID                         (Title
Line  3:      5  14 18                            (NATOMS, NORBS, NELECS
Line  4:   0.0000    0.0000    0.0000             (Cartesian coordinates
Line  5:   1.2270    0.0000    0.0000
Line  6:   1.9164    1.1676    0.0000
Line  7:   1.3973    1.9625    0.0000
Line  8:   1.8882   -0.8836    0.0000
Line  9:    1 2 1 3 3                             (Labels
Line 10:    8 6 1                                 (Atomic Numbers
Line 11:    2 2 1                                 (Princ. Quant. Nos.
Line 12:    4 4 1                                 (Orbitals per atom
Line 13:   2.699905  2.699905                     (Exponents
Line 14:   1.787537  1.787537  
Line 15:   1.331967                    (below: eigenvector coefficients
Line 16:   -0.1616  0.4769  -0.0514   0.0000  -0.0628  -0.3488  -0.0032  
Line 17:    0.0000  0.2634   0.6282  -0.0893   0.0000  -0.2925  -0.2457  
Line 18:    CENTER=2 LINE=(0,0,1) EDGE=4          (Rest of data
.el
 After all these data are read in, the program continues as usual, but many
limitations apply. For example, if M.O. was used, then the key-words PSI=,
HOMO and LUMO would all give the same result.
These limitations are due to the absence of other data.
.page
 The second set of key-words defines the type of plot to be drawn; much
of this data is obligatory in that there are no defaults for certain data.
The user must supply:
.ls
.le;A definition of the center of the plot.
.le;A definition for the axis of the vector perpendicular to the plot.
.le;A length, in Angstroms, of the edge of the plot.
.x EDGE
.els
 Other data, which are mutually exclusive, are:
.ls
.le;A molecular orbital number, given by PSI=n, or
.le;A highest-occupied molecular orbital, defined by the key-word HOMO, or
.le;A lowest-unoccupied molecular orbital, defined by the key-word LUMO, or
.le;The explicit molecular-orbital occupancy, see key-word OCCUPANCY, or
.le;The normal electron density of the molecule. This is specified by
the absence of the other key-words; i.e., this is the default.
.els
 If a full electron density map is specified. (That is, if the last option 
above is taken), then another key-word, BONDS, can be used. BONDS will remove the
electron density due to the atom from the plot. The effect of this is to 
generate a plot showing where the charge build-up in the bonds occurs.
.br
Note: Ionic separations are not shown by BONDS.
.page
.hl Full List of Key-Words
.spacing 2
.x Key-words, full list of

.lt
 ADD         -  The current plot is to be added to the next plot.
 AXIS=       -  Angle of plane of plotted square to viewer's eye.
 BONDS       -  Subtract the electron density due to the atoms.
 CENTER=     -  Used to define the center of the plot.
 EDGE=       -  Definition of the length of one side of the plot
 FINE        -  Produce four times the default number of contours.
 GRID        -  Overlay x-y grid onto plot. Used with AXIS.
 HOMO        -  Plot the highest occupied M.O.
 LINE=       -  Used to define the axis perpendicular to the plot.
 LUMO        -  Plot the lowest unoccupied M.O. 
 MULT=       -  Multiply vertical relief by specified amount.
 OCCUPANCY   -  Used to allow the individual M.O. populations to be specified.
 PSI=        -  Plot a specific M.O. 
 PHASE       -  Multiply the current plot by a constant.
.el
.spacing 1

.page
.hl Definitions of Key-Words
.c
 ADD
.x ADD
 When two or more pictures are to be added together to form a single picture,
ADD must be used. For example, a user might want to add or subtract
two M.O.s to see the effect. Thus if M.O. 2 and M.O. 3 were to be
added together, the data for such a calculation could be:
.lt
Line 1:
Line 2:   FORMALDEHYDE
Line 3: CENTER=2 LINE=(0,0,1) EDGE=2 PSI=2 ADD
Line 4: CENTER=2 LINE=(0,0,1) EDGE=2 PSI=3 
Line 5:
.el
 See also PHASE
.c
 AXIS=
.x AXIS, Definition of
 A square plot is drawn by default. If the user wants to show the
three-dimensional nature of the contour map a facility exists to
tilt the plot so that a foreshortened and rotated plot is drawn.
 Of course, the degree of tilt depends very much on the system;
there is no facility in DENSITY to eliminate "hidden lines", so
the tilt should not be so great that hidden lines would show.
 The range of n.nn in  AXIS=n.nn is 1.0 to 0.0. 1.0 (the default)
would give
a square plot, as if the user was viewing the plot from directly
overhead looking straight down; 0.0 gives a view of the plot as if the
user was looking at it from the horizon, looking horizontally.
 Clearly, AXIS=1.0 would not show the relief. If GRID were
used only a perfectly featureless square grid would be seen.
Conversely, AXIS=0.0 would show the contours as
perfectly flat, straight lines. Therefore, a better choice would
be AXIS=0.6.
.s 2
.tp 15
.c
 BONDS
.x BONDS, Definition of
 Not to be used in conjunction with HOMO, LUMO, PSI=, or OCCUPANCY.
 BONDS calculates the average atomic orbital occupancy of each atom, 
and subtracts that number from the density matrix. The result is a plot
whose average value is zero, and which shows where the electrons have come from and
gone to when the bonds are formed. 
.s 2
.tp 16
.c
 CENTER=
This key-word is essential. It defines the center of the plot (see BACKGROUND).
Two formats are provided to define the center:
(a) an atom number can be used, and (b) an absolute cartesian coordinate 
can be specified.
 (a) Definition by atom number.

 Format: CENTER=n  The location of atom n is defined as the center of 
.x CENTER, Definition of
the plot. Thus if atom
n has cartesian coordinates (x=0.5, y=1.4, z=-0.8) then the center of the
plot is (x=0.5, y=1.4, z=-0.8). Dummy atoms are not counted, so if
any dummy atoms were used in the definition of the geometry, they must
be ignored when considering the atom number. In other words, only a real
atom can be used to define a point in the molecule.
 (b) Definition by Absolute Cartesian Coordinate.
 Format: CENTER=(n.nn,n.nn,n.nn) The location of the center of the plot
is defined as (n.nn,n.nn,n.nn). Of course, before such a center can
be defined, the user must know the cartesian coordinates of the atoms in
the molecule.
 Irrespective of which option is used, the center of the plot will be
converted internally into absolute cartesian coordinates.
.s 2
.tp 14
.c
 FINE
.x FINE, Definition of
 Normally, between 10 and 25 contours are plotted. In order to increase
this number, FINE can be used, in which case 40 to 100 contours
will be generated.
.s 2
.tp 13

.c
 EDGE=n.nn
.x EDGE, Definition of
 The length of one side of the graph-plot is defined as being n.nn 
Angstroms.
.s 2
.tp16
.c
 GRID
.x GRID, Definition of
 GRID can be used to enhance the legibility of a plot. It draws
a regular mesh of lines across the plot, with the same relief as the
contours. GRID should only be used with an AXIS which is not
1.0.
.s 2
.tp 16
.c
 HOMO        
.x HOMO, Definition of
 For closed-shell systems with non-degenerate highest occupied molecular
orbitals, the key-word HOMO can be used to produce an intensity
map of the highest occupied molecular orbital. For other systems, the
key-word PSI= should be used.
.s 2
.tp 16
.c
 LINE=
This key-word is essential. It defines the vector perpendicular to the plane
of the plot (see BACKGROUND). Two formats are provided to define the axis
perpendicular to the plane of the plot; these formats use radically different
concepts, so users are cautioned to verify that they understand both
definitions and the distinction between them.
(a) an atom number can be used, and (b) an absolute cartesian coordinate 
can be specified.
 (a) Definition by atom number.
 Format: LINE=n  The axis of the plot is defined by the vector drawn from 
atom n to the defined center of the plot.
.x LINE, Definition of
Thus if atom
n has cartesian coordinates (x=0.5, y=1.4, z=0.2) and the center of the plot
is at point (x=0.5, y=1.4, z=-0.8) then the axis of the plot
is (0.0, 0.0, 1.0). Dummy atoms are not counted, so if
any dummy atoms were used in the definition of the geometry, they must
be ignored when considering the atom number. In other words, only a real
atom can be used to define the axis of the graph.
 (b) Definition by Absolute Cartesian Coordinate. The vector need not be 
normalized, but must not be of zero length.
 Format: LINE=(n.nn,n.nn,n.nn) The axis of a line perpendicular to the
plane of the plot is (n.nn,n.nn,n.nn). This axis need not be normalized,
but must be finite; that is, the only axis not allowed is (0,0,0)
 Irrespective of which option is used, the axis of the plot will be
converted internally into a unit vector in cartesian coordinates.
.s 2
.tp 16
.c
 LUMO
.x LUMO, Definition of
 For closed-shell systems with non-degenerate lowest unoccupied molecular
orbitals, the key-word LUMO can be used to produce an intensity
map of the lowest unoccupied molecular orbital. For other systems, the
key-word PSI= should be used.
.s 2
.tp16
.c
 MULT
.x MULT, Definition of
 There is a default scale for the relief of a plot, when viewed as a
3-D structure. If this default is not suitable, say the plot is too flat,
then MULT=n.nn can be used to change the vertical scale. MULT=1.0 will
do nothing, MULT=2.0 will increase the vertical relief.
.s 2
.tp 16
.c
 OCCUPANCY   
.x OCCUPANCY, Definition of
 When the user wants to explicitly define an electronic configuration
for a system, OCCUPANCY is used. This key-word requires the explicit
occupancy of the M.O.s to be defined on the next line, in I1 format.
For example, to define methane with one electron in each of the three
triply-degenerate levels, the lines
.tp 12
.lt

Line 1:  CENTER=1 LINE=(0.0, 0.0, 1.0) OCCUPANCY EDGE=2.0 
Line 2:2111
.el
could be used. Similarly, if an excited ethoxy radical were to be specified,
the following definition could be used:
.tp 12
.lt

Line 1:  CENTER=(0.7,0.0,0.0) LINE=(0.0, 0.0, 1.0) OCCUPANCY EDGE=2.0 
Line 2:22222201
.el
.s 2
.tp 12
.c
 PHASE
.x PHASE, Definition of
 Two formats for this are possible. If PHASE is specified, the
data being added to the plot is multiplied by -1, i.e. negated. 
If PHASE=n.nnnn is used,
the data are multiplied by n.nnnn. For example, to form the normalized
M.O. of formaldehyde resulting from equal and opposite amounts of the
second and third M.O.s, i.e. 0.7071(M.O.2 - M.O.3), the following 
data could be used.
.lt
Line 1:
Line 2:   FORMALDEHYDE
Line 3: CENTER=2 LINE=(0,0,1) EDGE=1 PSI=2 PHASE=0.7071 ADD
Line 4: CENTER=2 LINE=(0,0,1) EDGE=1 PSI=3 PHASE=-0.7071
Line 5:
.el
 Note that the first phase is positive. It is not the plot that is reversed,
only the data going in to the plot, so that the data arising from line 4
are to be multiplied by -0.7071 before being added to the plot.
.s 2
.tp 12
.c
 PSI=        
.x PSI, Definition of
 A specified molecular orbital is to be plotted. Note that degenerate
M.O.s cannot be unambiguously represented, as they are ill-defined
by a random unitary transform. To plot degenerate M.O.s the user should
be prepared to use PHASE and ADD.
.s 2
.tp 12
.c
TIGER
.x TIGER, Definition of
 A file suitable for direct submission to a Paper Tiger is generated
by specifying TIGER. This is VAX-specific and will need to be
modified to run on other machines.
.ch BACKGROUND
.LEFT MARGIN 10                 ! One inch

.hl Definition of the area of the graph plot.
 The graph plot is a square, which represents a slice or cut through the
.x Plot, area, definition of
volume of a molecule. To define a square, seven numbers are needed. In order,
these are:
.ls
.le;Three numbers to define the position of the center of the square.
.le;Three numbers to define an axis perpendicular to the surface of the square.
.le;One number to define the length of the edge of the square.
.x EDGE
.els
 To specify the center of the square, the key-word CENTER is provided. CENTER
can be either an atom number, e.g. CENTER=4, or an absolute cartesian 
coordinate, e.g. CENTER=(0.6,0.5,2.4). If an atom is used, the absolute
cartesian coordinates of that atom are used to define the center
of the square. In either case, the center
of the square, i.e. the intersection of the two diagonals of the square,
is defined by the resulting absolute cartesian coordinate. 
 Examples:
.ls
.le;Methane, center of plot to be on one of the hydrogen atoms, these 
.x CENTER, Methane, example of
being atoms 1, 3, 4, and 5. Use CENTER=1 or CENTER=3 etc.; as an 
alternative, assuming the first atom is at the origin, CENTER=(0.0,0.0,0.0)
could be used.
.TP17
.le;Benzene, given cartesian coordinates as follows
.lt
             X         Y         Z
  C1       0.0000    0.0000    0.0000
  C2       1.4066    0.0000    0.0000
  C3       2.1099    1.2182    0.0000
  C4       1.4066    2.4363    0.0000
  C5       0.0000    2.4363    0.0000
  C6      -0.7033    1.2182    0.0000
  H1      -0.5451   -0.9442    0.0000
  H2      -0.5451    3.3805    0.0000
  H3       1.9517   -0.9442    0.0000
  H4      -1.7935    1.2182    0.0000
  H5       3.2002    1.2182    0.0000
  H6       1.9517    3.3805    0.0000
.el
then, if a plot of benzene with the center of the molecule at the middle
of the picture, CENTER=(0.7,1.2,0.0) could be used.
.x CENTER, Benzene, example of
.x Benzene, example of CENTER
.le;Benzene, pi-system. Using the above cartesian coordinates for the
benzene ring, the center of the plot must be above the plane of the
.x CENTER, Benzene, pi-systems
.x Pi-systems, in benzene
ring in order to avoid the nodes in the pi-system. In this case
CENTER=(0.7,1.2,0.5) would be appropriate.
.els
 Once the center of the plot is defined, the orientation of the plane
of the plot needs to be specified. Thus if a plot of methane was wanted,
and the center of the plot was on the carbon atom, the following 
cross-sections could be drawn:
.ls
.le;Four-fold symmetry: Plot is oriented in the reflection plane of the
.x CENTER, examples of
SU4D symmetry operation. To specify this, absolute cartesian
coordinates would need to be used.
.le;Three-fold symmetry: Axis of plot is from carbon to any hydrogen.
To specity this, LINE=4 could be used.
.le;Showing two C-H bonds: Given that the first three atoms are H, C, and
H, then the axis (0.0,0.0,1.0) would be perpendicular to the plane
of  these three atoms (assuming that all three atoms had the same z component).
.els
.TP 12
 It might be helpful in visualizing the axis of the plot by imagining the plot
without the line specified. The center of the plot is already defined, but
the square is still free to rotate in all three dimensions, so only one
point in the square is defined. The plane of the plot is then defined by LINE.
.TP 12
 The final unknown is the size of the plot. This is specified by EDGE=n.nn.
The length n.nn defines the length, in Angstroms, of the side of the plot.
Note that most molecules are bigger than their simple atomic coordinates
would suggest. As an example, to get a benzene ring to fit inside a plot,
EDGE=6.0 would be needed, even though the H-H distance across the ring is
only 4.8-5.0 Angstroms.
.tp 17
.s 1
.hl CONTOURS
.x Contours, values of
 There are normally between 10 and 25 contours. The contours are separated
by steps of size 1, 2.5, and 5 times 10 to some integer power. The most
common step sizes are:
.lt
 0.0010, 0.0025, 0.0050, 0.0100, 0.0250, 0.0500, 
 0.1000, 0.2500, 0.5000, 1.0000, 2.5000, 5.0000
.el
in units of electrons per cubic Angstrom, for electron density, and
the square root of these units for wave-functions.
 The flexible nature of the step size means that the user can expect to see
something in the picture, but close attention should be paid to the contour
interval. If the step size is excessively small, less than 0.00001, 
the user will be warned, but the calculation will be continued.
.tp 17
.s 1
.hl Density Around Atoms
 All Slater atomic orbitals have a node at the nucleus, with the important exception
of Hydrogen. One result of this is that hydrogen is the only element whose
electron density has a maximum at the nucleus. All other atoms have an almost
zero electron density at their nucleus, this is usually disconcerting 
at first sight,
but is a consequence of the behavior of Slater-type orbitals.
 For planar systems, it is a good idea to plot the density in the plane
a fraction of an Angstrom above the plane of the molecule.
.s 1
.tp16
.hl Difference Maps
 A difference map shows where electron density builds up when bonds are
formed. It will NOT show ionic character. To form a BONDS map
.x Difference Maps
.x Maps, difference
the electron density arising from the atoms is spherically averaged and
then subtracted from the density matrix. For example, suppose the atomic
orbital populations on an atom arising from the un-renormalized eigenvectors 
are  1.0, 1.1, 1.2, and 1.3 for s, px, py, and pz, respectively. Then the
spherically averaged populations are 1.0, 1.2, 1.2, and 1.2 for these atomic
orbitals. 
.s 3
.tp 15
.hl Renormalization of Molecular Orbitals
  All maps are calculated from data originally derived from the supplied
.x Renormalization of Eigenvectors
molecular orbitals. These M.O.s are presumed to be un-renormalized: that
is, the sum of the squares of the coefficients of the M.O.s add to unity.
The first step, therefore, is to re-normalize the M.O.s. This is done
by matrix multiplying the starting eigenvector matrix by the 
inverse-square-root of the overlap matrix. This requires the presence of
the inverse-square-root of the overlap matrix, which must be supplied by
the MOPAC file. If re-normalization is not done, the total 
electron density would be too large, and the difference maps would be
quite incorrect.



.ch Programming Considerations
.LEFT MARGIN 10                 ! One inch
 DENSITY is not intended to be a "stand-alone" program. Its main purpose
is to be run using data generated by an earlier MOPAC calculation.
Results from DENSITY need some further processing before a hard-copy
graph-plot is obtained.
 In order, then, the unformatted data required from 
a MOPAC type calculation, to be used as auxilliary 
data for DENSITY are as follows:-
 Record:
.lm 18
.ls
.le;Number of Atoms, Number of Orbitals, Number of Electrons,
X-Cartesian coordinates for all atoms, Y-Cartesian coordinates for all atoms,
Z-Cartesian coordinates for all atoms.
.le;Stopping and starting orbital counters for all atoms.
.le;S-orbital exponents for all atoms, p-orbital exponents for all atoms, 
d-orbital exponents for all atoms. Atomic numbers for all atoms.
.le;All the un-renormalized eigenvectors, atomic orbital indices running
fastest.
.le;Lower half triangle of the inverse-square-root of the overlap matrix.
.els
.lm 10
 DENSITY produces a default file suitable for a plotting program to use. 
This file has the following format:
 Line 1: This line consists of the single, unchanging, message
.lt

                    "NUMBER OF CONTOURS =  nn"
.el
 nn is the actual number of different contours supplied. The programmer
can use this message to detect the start of a contouring file, and if
desired, make use of the number nn to decide what action to take.
 Lines 2-4: Text. The programmer should print these lines without assuming
that the text makes any sense.
 Lines 5 on: The layout of these lines is as follows:-
.lt
<x-coordinate> <y-coordinate> <Pen code>  <Contour Height>
.el
As the whole plot is within the range 0.0 to 1.0, the x- and y-coordinates 
are fractions of unity. The pen code is integer, 1 for pen up, 2 for pen down.
The contour height is real, and self-explanatory.
 Near to the end of the file is a single contour of height 99.999. This "contour"
draws a box around the plot.
 Finally, the last few lines are the chemical symbols. These use the
same layout as Lines 5 on, but in place of the pen code, the chemical symbols
are present.
 DENSITY can also produce a byte-structured file suitable for use
on a Paper Tiger printer. This option is activated by the key-word TIGER
on the first line of the data-file. If neither of these outputs are
suitable, the programmer should modify subroutine PLOTGR.
.CH DESCRIPTION OF PROGRAM
.LEFT MARGIN 10                 ! One inch
.HL 1 Fortran Files
 DENSITY consists of several FORTRAN-77 files, these are:-
.x FORTRAN files, list of
.ls
.le;CNTOUR: A general 2 or 3-dimensional contour-drawing subroutine.
.le;DRAW: A subroutine to draw one contour.
.le;EULER: Rotates a point in 3-D space.
.le;DATIN: Reads in data defining one particular picture.
.le;MAIN: Main segment; generates array for contouring.
.le;MAXMIN: Calculates highest and lowest points in the picture.
.le;PLOTGR: Interface between FORTRAN-77 and graphics routine.
.le;READIN: Gets in data defining molecule
.le;PTCLN, PTCLR, PTDOT, PTLABL, PTLINE, PTOPEN, PTOUT: All Paper-Tiger
routines.
.le;PLOTGR: Generates data for graph plotter.
.els
.tp 17
.hl 1 Description of FORTRAN Files.
.s 4
.tp 12
.c
 CNTOUR
.x Subroutine CNTOUR
.x CNTOUR, Subroutine
 CNTOUR is a general contour-drawing routine. It calls subroutines
DRAW, EULER, and MAXMIN. The contour map can be tilted to show the
three-dimensional structure of the density map.
.tp 12
.s 3
.c
DRAW
.x DRAW, Subroutine
.x Subroutine DRAW
 Draws a single contour, given a starting address. Called by CNTOUR
to allow
users to rapidly write their own interfaces. 
.tp 12
.s 3
.c
EULER
.x EULER, Subroutine
.x Subroutine EULER
 Performs a three-dimensional rotation on a three-dimensional
point provided; the resulting point is then used as a data
point for plotting. 
.tp 12
.s 3
.c
DATIN
.x DATIN Subroutine
.x Subroutine DATIN
 All the user-supplied data defining a picture are read in in DATIN.
Some checking is done to ensure that the key-words provided are
compatable.
.tp 12
.s 3
.c
MAIN
.x MAIN, Program
.x Program MAIN
 The main program, generates the wave-function in terms of atomic
orbitals, this is then used to calculate the pixel array, which in
turn is used by the contouring subroutine.
.tp 13
.s 3
.C
 MAXMIN
.x MAXMIN, Subroutine
.x Subroutine MAXMIN
 Prints the values and positions of the maxima and minima on the plot.
No data are altered by a call to MAXMIN.
.tp 12
.s 3
.c
PLOTGR
.x PLOTGR, Subroutine
.x Subroutine PLOTGR
 The interface between FORTRAN-77 and the graphics routines is provided
by PLOTGR. Only very simple graphics commands are used, in order to
permit easy conversion of the program to run on other machines.
Programmers will need to change PLOTGR to suit local equipment.
.tp 12
.s 3
.c
PTxxx
.x PTxxx, Subroutine
.x Subroutine PTxxx
 The set of subroutines PTCLN, PTCLR, PTDOT, PTLABL, PTLINE, PTOPEN,
and PTOUT are VAX-specific, and are used to generate a bit-map of the
picture. This is suitable for submission to a Paper Tiger printer.
 Users of machines other than a VAX must modify or remove these
subroutines. (It is probably best to modify PLOTGR.)
.CH INSTALLING DENSITY

.spacing 2

.x Installing DENSITY
.x DENSITY, installing
  DENSITY is distributed on a magnetic tape as a set of FORTRAN-77 files,
along with ancillary documents such as command, help, data and results files.
The format of the tape is that of the VAX-11/780 computer. The following
instructions apply only to users with VAX computers; users with other machines
should use the following instructions as a guide to getting DENSITY up and
running.
.ls
.le;Put the magnetic tape on the tape drive, making sure it is write protected.
.le;Allocate the tape drive with a command such as $ALLOCATE MTA0:
.le;Go into an empty directory which is to hold DENSITY
.le;Mount the magnetic tape with the command $MOUNT MTA0: DENSIT
.le;Copy all the files from the tape with the command $COPY MTA0:_*._* _*/LOG
 If this command does not work, try $COPY/LOG MTA0:_*._* _*
.els
 A useful operation after this would be to make a hard copy of the
directory.  You should now have the following sets of files in the directory:
.ls
.le;A set of FORTRAN-77 files, see Appendix 1.
.le;The command files COMPILE, DENSITY, and RDENSITY.
.le;A help file called DENSITY.HLP.
.le;A text file DENSITY.MAN.
.le;Some test-data files, and results files.
.els
.page
.c
STRUCTURE OF COMMAND FILES
.c
COMPILE
.x Command Files, COMPILE
.x COMPILE, COM file
 The parameter file DIMSIZES.DAT should be read and, if necessary, modified
before COMPILE is run.
 COMPILE should be run only once. It assigns DIMSIZES.DAT,
the block of FORTRAN which contains the PARAMETERS for the 
dimension sizes to the logical name "SIZES". This is a 
temporary assignment, but the user is strongly advised to make it 
permanent by suitably modifying LOGIN file(s).
 All the FORTRAN files are then compiled, using the array sizes given in
DIMSIZES.DAT. These should be modified before COMPILE is run. If, for 
whatever reason, DIMSIZES.DAT needs to be changed, then COMPILE should be
re-run, as modules compiled with different DIMSIZES.DAT will be 
incompatible. 
 The two parameters within DIMSIZES.DAT that the user can modify are
MAXLIT and MAXHEV. MAXLIT is assigned a value equal to the largest number
of hydrogen atoms that a DENSITY job is expected to run, MAXHEV is assigned the
corresponding number of heavy (non-hydrogen) atoms. 
 This operation takes about 2 minutes, and should be run "on-line", as a
question and answer session is involved.
 When everything is successfully compiled, the object files will then be
assembled into an executable image called DENSITY.EXE. Once the image exists, 
there is no
reason to keep the object files, and if space is at a premium these can
be deleted at this time. They should be kept if modifications are to be
made to the program, in order to allow rapid re-linking with the modified
subroutines.
.tp 16
.c
DENSITY
 This command file submits a DENSITY job to the FLASH queue. 
Before use DENSITY.COM
.x DENSITY, Command files
should be modified to suit local conditions. The user's VAX is assumed
to use a queue called FLASH.
.tp 16
.c
RDENSITY
  RDENSITY is the command file for running DENSITY. It assigns all the data
files that DENSITY uses to the channels.
.x RDENSITY, Command Files
If the user wants to use other
file-name endings than those supplied, the modifications should
be made to RDENSITY.
.tp 20
  A recommended sequence of operations to get DENSITY up and running would be:
.no fill
 (1)  Modify the file DIMSIZES.DAT; the default sizes are 20 heavy atoms 
      and 20 light atoms.
 (2)  Read through the COMMAND files to familiarize yourself with what 
      is being done.
 (3)  Edit the file  DENSITY.COM to use a local queue name, if any.
 (4)  Edit the file RDENSITY.COM if the default file-names are not 
      acceptable.
 (5)  Edit the login command file to insert the following lines:
             $ASSIGN [DENSITY] DENSITYDIRECTORY             (Note 1)
             $ASSIGN DBA0:[DENSITY]DENSITY HLP$LIBRARY      (Note 2)
             $DENSITY  :== @DENSITYDIRECTORY:DENSITY          (Note 3)
                 (look at LOGIN.COM at this point.)
         Note 1: Substitute the actual name of the directory which will
                 hold DENSITY if the name is not to be [DENSITY]. If you cannot identify the
                 directory, remove all references to DENSITYDIRECTORY from
                 the COMMAND files.
         Note 2: This allows HELP to access the DENSITY help library. If
                 HLP$LIBRARY is already assigned, then use the first 
                 empty library, HLP$LIBRARY__n. The assignment may have 
                 to be modified (Consult someone who understands this!).
         Note 3: This assigns the word DENSITY to the command file 
               DENSITY.COM.
 (6) Execute the modified LOGIN command so that the new commands are
      effective.
  (7) Run COMPILE.COM. This takes about 2 minutes to execute.
  (8) Enter the command
.fill
$DENSITY 
      You will receive the message
 "What file? :" 
      to which the reply should be the actual data-file name. For
      example, "ETHYLENE", the file is assumed to end in .GRA, 
      e.g. ETHYLENE.GRA.

  (2) Run the (supplied) test molecules, and verify that DENSITY is 
     producing "acceptable" results.
  (3) Make some simple modifications to the datafiles supplied in 
     order to test your understanding of the data format
  (4) When satisfied that DENSITY is working, and that data files can 
     be made, begin production runs.
.FILL
.page
How to use DENSITY

The COM file to run the new program can be accessed using the command
"DENSITY" followed by none, one or two arguments.  Possible options are:
.NO FILL
DENSITY   MYDATAFILE 
.FILL
.B
DENSITY
The main files that are required by DENSITY are
.NO FILL
                                                  
           File          Channel     Description
         <filename>.GPT    13    MOPAC generated file.
         <filename>.GRA     5    User-generated data file
.FILL

The main files that are produced are:
.NO FILL
                   
           File          Channel     Description
        <filename>.OUT      6    Results
        <filename>.TEC     15    Formatted file of graph-plot - see DRAW
        <filename>.PLT     11    Byte file for Paper Tiger
.FILL

.AX INDEX
.LEFT MARGIN 10                 ! One inch
.px

