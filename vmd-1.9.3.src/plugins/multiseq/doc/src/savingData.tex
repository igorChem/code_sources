\label{unit9}

% Saving Data from MultiSeq sessions
%We just need to figure out what we're doing with this
\section{Working With Sessions}
The \textsf{Load} and \textsf{Save Session} options from the
\textsf{File} menu provide a way to save and load all of the files,
alignments, and visual representations currently in use within MultiSeq
in a convenient package.  

\subsection{Save Session}
You can save a session of MultiSeq, with all of the files, alignments,
and visual representations, by simply going to the \textsf{File Menu}
and selecting \textsf{Save Session}.  You will be prompted to save the
session, and will have the opportunity to create a unique name for the
session here.  Hit the \textsf{OK} button.  A file will be generated
with a \texttt{.multiseq} extention along with a directory filled with
various files necessary to load the saved session into MultiSeq.  Please
note that both the generated file and directory have to be in the same
directory location in order to load up the session in the future
properly.  

\subsection{Load Session}
Unlike \textsf{Import Data} (also in the \textsf{File} menu),
\textsf{Load Session} opens up a previous session of MultiSeq with all
of the sequence and structure files aligned, and using previous coloring
and drawing methods.  To load a previously saved MultiSeq Session,
simply select the \textsf{File} menu and \textsf{Load Session}.  A file
broswer will appear allowing you to select a file with the extension
\texttt{.multiseq} and make sure it has a corresponding directory of the
same name.

\section{Other Ways To Export Data}

\subsection{Save to PostScript}  From the \textsf{File} menu, if you
choose \textsf{Save Screenshot}, you will be able to save a postscript
version of the MultiSeq window.
%\subsection{Write PDB from selection...}
%During a MultiSeq session, you  may want to save various persepectives
%of the structural alignments you created.  Often these images are
%generated by highlighting specific portions of the aligned protein
%sequences. If you would like to study your selections further, you can
%can do so by generating your own PDB file(s).  To begin this process:
%\begin{enumerate}
%\item Highlight the portions of the sequence that you want to examine in
%the Sequence Display of the MultiSeq window.
%\item In the same window top pull-down menu, go to View$\rightarrow$
%Highlight style. Multiple Highlight styles will appear to choose from.
%Select one and make sure it appears in the OpenGL Display.
%\item Click on File $\rightarrow$Write PDB from selection....
%\item The PDB file(s) will be saved in a directory that can be chosen by
%clicking on the File $\rightarrow$ Choose Work Directory.... If you
%have not selected your Work directory, you be prompted to choose a
%directory when you click on Write PDB from selection....
%\end{enumerate}
%\subsection{Save work data into another directory...}


