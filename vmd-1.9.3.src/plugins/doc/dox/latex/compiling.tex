\section{Compiling plugins from source code}\label{compiling}


 Compiling the plugin tree involves following a few simple steps: \begin{CompactItemize}
\item 
Getting the source code (via CVS, or from a VMD source distribution) \item 
Compiling an appropriate Tcl if it is not already available on your computer system. \item 
Compiling the plugin source tree  \item 
Generating a plugin distribution directory, including documentation. \end{CompactItemize}
\section{Libraries required by plugins}\label{pluginfiles}
 

 Before you can successfully build plugins, several other things must be compiled and installed. At a bare minimum, a normal plugin build requires {\tt Tcl} to be compiled  and linkable from within the directories referenced by the  TCLINC and TCLLIB environment variables. Without these libraries and their associated header files,  compilation of the plugin source code will fail.



 If you wish to build the plugins for AMBER 9, and MMTK Net\-CDF trajectories, then the NETCDFLIB and NETCDFINC environment variables must be set. In order to link the static plugins with VMD, Cat\-DCD or other software packages, you'll need to remember to add the Net\-CDF library to the linkage flags for the executable being build. In the case of Cat\-DCD this is done automatically by the Make-arch script. For VMD, you will need to enable the \char`\"{}NETCDF\char`\"{} configure option. For other packages, you'll want to add the same NETCDFLDFLAGS macro used in Make-arch to your link flags.

\section{Compiling the plugin source tree}\label{compilefromsrc}
 

 What follows is a simple example of how we compile one of the plugin architectures on our local machines here at UIUC.  Note that you must use GNU make to compile the VMD plugin tree as we use various Makefile features that are not provided by some Unix vendor implementations of Make:

\small\begin{alltt}
## MacOS X 10.2 framework paths used at UIUC
setenv TCLINC -I/Projects/vmd/vmd/lib/tcl/include
setenv TCLLIB -F/Projects/vmd/vmd/lib/tcl

cd /to/your/work/area/plugins/
gmake MACOSX TCLINC=$TCLINC TCLLIB=$TCLLIB/lib_MACOSX
\end{alltt}\normalsize 


\section{Customizing builds and concurrent compilation on multiple machines}\label{makeworld}
 

 If one wants to compile the plugin tree on multiple target platforms  concurrently, or customize the build script for ease of repeated  compilations, a shell scripta named {\bf build.csh} is provided which is invoked by the '{\bf make world}' build target. The build script recognizes machines by their hostname, and sets the TCLINC and TCLLIB environment variables appropriately, and runs the necessary build commands to build on as many target machines as is desired at the site. The VMD development team uses this script to automate compilation on all of the standard supported target platforms. You can edit build.csh by changing the  switch statement to recognize your local build machine or by replacing the settings in the \char`\"{}default\char`\"{} switch statement target with your own.

\section{Making a plugin distribution directory}\label{makedistrib}
 

 Once the plugin tree has been succesfully compiled, the next step is to build a distribution directory. This step populates a target directory with all of the compiled plugin binaries, as well as the platform-independent plugin scripts, documentation and other data files used by the installed  plugins. This step is the final step required before one can succesfully {\tt build VMD itself from source code},  as VMD uses the plugin distribution  directory as the source for {\bf vmdplugin.h} and other source files, as well as statically-linked plugin libraries. 

 Before you run the '{\bf make install}' command, you must first set the PLUGINDIR environment variable to the name of your distribution  target directory, using the appropriate shell command. Here's a  simple example: \small\begin{alltt}
cd /to/your/work/area/plugins/
setenv PLUGINDIR /final/desitation/directory/for/compiled/plugins
gmake distrib
\end{alltt}\normalsize 




 Assuming that everying goes as it should, the '{\bf make distrib}'  command will copy all of the required plugin binaries, support files, and  documentation to the target directory specified by the PLUGINDIR environment variable. Once this is done, you can proceed with  {\tt compiling VMD},  or using the final target directory to update your local installation of VMD's dynamically loaded plugins.

