\section{Introduction}\label{intro}




 The VMD plugin system is a relatively simple mechanism by which new functionality or updated code can be dynamically loaded from Tcl/Python scripts, shared object files (Unix-like systems),  or dynamic link libraries (Windows).



 The main idea behind the inclusion of plugins in VMD is to facilitate dynamic extensions to core parts of VMD at runtime without the need to recompile the entire program from scratch. This allows end-users to  modify VMD with new features, support for additional file formats,  and even new graphical user interfaces, all without having to change the main VMD source code. In addtion to increasing code modularity,  plugins provide a mechanism by which 3rd party developers can distribute extensions to the VMD user community at larger, without the need for  custom versions of VMD. Users can continue to use the same version of VMD they have installed, but gain new features through the installation of new plugins. This capability allows developers to release plugins which support new computational chemistry file formats, in-between major releases  of VMD. This is extremely helpful since new versions of popular chemistry packages are release frequently, and rarely at the same time that new  versions of VMD are released.



 VMD currently supports several types of plugins: \begin{CompactItemize}
\item 
{\bf Molecule file reader plugins} {\rm (p.\,\pageref{molfileplugins})} (natively read one or more file formats) \item 
Molecule file translator plugins (translate to another supported format) \item 
{\bf Tcl/Tk packages} {\rm (p.\,\pageref{tcltkplugins})} and  {\bf Python modules} {\rm (p.\,\pageref{pythonplugins})} which may optionally be added to  the VMD \char`\"{}Extensions\char`\"{} menu. \end{CompactItemize}


 By far the simplest type of plugins to develop are Tcl packages and Python modules. These plugins can be anything from a simple script that simplifies a commonly performed set of calculations to something more sophisticated that works with other plugins to perform more complex tasks such as structure building or running calculations with external electrostatics or molecular dynamics engines. Tcl packages and Python modules are easier to distribute than other plugins since they don't  normally have any compiled components, so they can be very easily  developed, tested, and deployed. Carefully written plugins of this type can be made to work with several versions of VMD as long as no version-specific features or text commands are used by the plugin.



 Molfile file plugins are normally implemented with C/C++ code that is compiled to statically and dynamically loadable libraries. The static plugins are used when compiling a particular VMD binary  distribution as a fall-back and to provide a minimum level of built-in functionality. New platforms typically support static plugins first and dynamic plugins once shared library loading code is written. Molfile plugins are more challenging to write, debug, and deploy than the Tcl or Python plugins since they have to be compiled separately for each target operating system and platform combination. VMD includes mechanisms for dynamically loading new plugins in place of statically linked plugins and older revs of dynamically loaded  plugins. File plugins are implemented with compiled C/C++ primarily for the increased efficiency in performing byte-level file and memory manipulations and operations on large multidimensional arrays. At some point we expect to provide scripting-based methods for loading files though this isn't presently available. One could certainly implement a molfile plugin using one's favorite scripting interface as long as it could provide the necessary C callbacks, data structures, etc.

