\documentclass[a4paper, DIV12]{scrartcl}

\usepackage{hyperref}
\usepackage{xspace}
\usepackage{verbatim}           % required for \verbatim and \endverbatim
\usepackage{fancyvrb}
\usepackage{longtable}

% no graphics so far
%\usepackage{graphicx}

\newcommand{\ie}{\emph{i.e.}\xspace}
\newcommand{\eg}{\emph{e.g.}\xspace}
\newcommand{\etal}{\emph{et al}\xspace}
\newcommand{\pbctools}{PBCTools\xspace}

\hypersetup{
  pdfauthor={Olaf Lenz}, 
  pdftitle={PBCTools Plugin User's Guide}
}

\RecustomVerbatimEnvironment{Verbatim}{Verbatim}{frame=single,commandchars=\\\{\}}

\begin{document}
\title{\pbctools Plugin User's Guide}
\author{Jerome Henin \and Olaf Lenz \and Cameron Mura \and Jan Saam}
\date{Version 2.8}

\maketitle

The plugin \texttt{pbctools} provides Tcl functions to handle periodic
boundary conditions. 

\section{Basic usage}
All of the plugin's functions can be accessed via the Tcl text command

\begin{Verbatim}
  pbc \textrm{\textit{subcommand}} [\textrm{\textit{options}}]...
\end{Verbatim}

\noindent that you can write in a VMD-Tcl-script or interactively
enter in the VMD console window or the VMD TkConsole (accessible via
VMD Main Menu $\rightarrow$ Extensions $\rightarrow$ Tk Console).
When no \textit{subcommand} is provided, a short help message will be
printed. The list of available subcommands can be found in table
\ref{tab:commands}.

\begin{table}[p]
  \centering
  \begin{tabular}{|p{0.3\textwidth}|p{0.55\textwidth}|r|}
    \hline
    \textbf{Subcommand} & \textbf{Description} & \textbf{p.}\\\hline\hline
    
    \mbox{\texttt{set} \textit{cell}} [\textit{options}\dots]
    & Set the VMD unit cell properties (\eg to use VMD's feature that
    allows to display periodic copies of the system). 
    & \pageref{sec:set}
    \\\hline
  
    \mbox{\texttt{get} [\textit{options}\dots]}
    & Get the VMD unit cell properties. 
    & \pageref{sec:get}
    \\\hline

    \mbox{\texttt{readxst} \textit{xstfile} [\textit{options}\dots]}
    & Read the VMD unit cell properties from an XST file.
    & \pageref{sec:readxst} 
    \\\hline

    \mbox{\texttt{writexst} \textit{xstfile} [\textit{options}\dots]}
    & Write the VMD unit cell properties to an XST file.
    & \pageref{sec:writexst} 
    \\\hline
  
    \texttt{wrap} [\textit{options}\dots]
    & When the atoms of the system are not all in one periodic image,
    but are distributed over various images, this function wraps all
    atoms into the chosen image. It is also possible to change between
    different representations of the unit cell (orthorhombic or triclinic).
    & \pageref{sec:wrap}
    \\\hline
  
    \texttt{unwrap} [\textit{options}\dots]
    & When overlong bonds (that stretch the whole system) occur and
    compounds (residues, segments, chains or fragments) are broken in
    the course of a simulation trajectory because atoms are wrapped
    around the periodic boundaries, this function will remove large
    jumps of atoms between consecutive frames.
    & \pageref{sec:unwrap}
    \\\hline
  
    \texttt{join} \textit{compound} [\textit{options}\dots]
    & When you have still broken compounds in frames after you have
    used \texttt{unwrap}, this function can be used to join broken
    compounds. Note, that this function is significantly slower than
    \texttt{unwrap}!
    & \pageref{sec:join}
    \\\hline
  
    \texttt{box} [\textit{options}\dots]
    & When you want to draw a box around the unit cell of your system,
    this function can be used. The box will automatically adapt to
    changes in the unit cell parameters in the course of a trajectory.
    & \pageref{sec:box}
    \\\hline
  
    \texttt{box\_draw} [\textit{options}\dots]
    & When the unit cell parameters do not change in the course of a
    trajectory, this function draws a static box that will not adapt to
    changes in the unitcell properties.
    & \pageref{sec:box_draw}
    \\\hline
  \end{tabular}
  
  \caption{List of the subcommands of the \pbctools plugin.}
  \label{tab:commands}
\end{table}

\section{Installation}

Since VMD version 1.8.6, the \pbctools plugin is part of the official
distribution of
VMD\footnote{\url{http://www.ks.uiuc.edu/Research/vmd/}}, and all
commands can be used within VMD without further preparation.

In the case that you are using an older version of VMD, or that you
want to use a more recent version of \pbctools than what came with the
VMD distribution, you can activate the \pbctools plugin as follows:

\begin{enumerate}
\item Fetch the \pbctools plugin from Github
\begin{Verbatim}
git clone git://github.com/olenz/pbctools.git
\end{Verbatim}
This will create a new directory \texttt{pbctools} in the current
directory.
\item Add the following lines to your VMD startup file
  (\verb!~/.vmdrc! on Unix or \verb!vmd.rc! on Windows)\footnote{For
    more details on the startup files, see chapter ``Startup Files''
    in the VMD User's Guide.}:
\begin{Verbatim}
set dir \textrm{\textit{pbctools-directory}}
source $dir/pkgIndex.tcl
package require pbctools
\end{Verbatim}
%$
\end{enumerate}


\newpage
\section{\texttt{set} -- Setting the unitcell parameters}
\label{sec:set}

To be able to work correctly, all other procedures of the \pbctools
plugin require the VMD unitcell parameters to be set.  Some file
formats and their readers provide the necessary information (\eg the
DCD, VTF and Amber crdbox formats).  When the format does not provide
the information, the parameters can either be set with help of the
command \texttt{pbc set}, or they can be read in from a file in XST
format via the procedure \texttt{pbc readxst} (see section
\ref{sec:readxst}).

\minisec{Syntax}
\mbox{\texttt{pbc} \texttt{set} \textit{cell}} [\textit{options}\dots]

\minisec{Description}

Sets the VMD unit cell properties to \textit{cell} in the specified
frames. \textit{cell} must either contain a single set of unit cell
parameters that will be used in all frames of the molecule, or it must
contain a parameter set for every frame.

\minisec{Example}

\begin{Verbatim} 
# set the unit cell side length to 10 in all frames
pbc set \{10.0 10.0 10.0\} -all
\end{Verbatim}

\minisec{Options}

\begin{tabular}{|p{0.3\textwidth}|p{0.64\textwidth}|}
\hline

\texttt{-molid} \textit{molid}$|$\texttt{top}
& Which molecule to use (default: \texttt{top}).
\\ \hline

\texttt{-first} \textit{frame}$|$\texttt{first}$|$\texttt{now}
& The first frame to use (default: \texttt{now}).
\\ \hline

\texttt{-last} \textit{frame}$|$\texttt{last}$|$\texttt{now}
& The last frame to use (default: \texttt{now}).
\\ \hline

\texttt{-all}[\texttt{frames}]
& Equivalent to \texttt{-first first -last last}.
\\ \hline

\texttt{-now}
& Equivalent to \texttt{-first now -last now}.
\\ \hline

\texttt{-namd}$|$\texttt{-vmd}
& Format of the unit cell parameters \textit{cell}. When \texttt{-vmd}
is used, a parameter set must be a list of the VMD unitcell parameters
\textit{a}, \textit{b}, \textit{c} (\ie the side lengths of the unit
cell) and optionally \textit{alpha}, \textit{beta} and \textit{gamma}
(the angles of the unit cell) for non-orthorhombic unitcells. When
\texttt{-namd} is used, a parameter set must contain the three unit
cell vectors \textit{A}, \textit{B} and \textit{C} (the 3D-vectors of
the unitcell sides) (default:\texttt{-vmd}).
\\ \hline

\texttt{-}[\texttt{no}]\texttt{alignx} 
& If the option \texttt{-namd} is used and the unit cell vector
\textit{A} is not parallel to the x-axis, \texttt{-alignx} will rotate
the system so that it is. If \texttt{-noalignx} is used, the function
will return with a warning when \textit{A} ist not aligned with the
x-axis.
\\ \hline
\end{tabular}
 


\newpage
\section{\texttt{get} -- Getting the unitcell parameters}
\label{sec:get}

\minisec{Syntax}
\texttt{pbc} \texttt{get} [\textit{options}\dots]

\minisec{Description}
Gets the VMD unit cell properties from the specified frames. Returns a
list of one parameter set for each frame or an empty list when an
error occured.

\minisec{Example}

\begin{Verbatim}
# get the unit cell parameters of the current frame
set cell [pbc get -now]
\end{Verbatim}

\minisec{Options}

\begin{tabular}{|p{0.3\textwidth}|p{0.64\textwidth}|}
\hline

\texttt{-molid} \textit{molid}$|$\texttt{top}
& Which molecule to use (default: \texttt{top})
\\ \hline

\texttt{-first} \textit{frame}$|$\texttt{first}$|$\texttt{now}
&  The first frame to use (default: \texttt{now}).
\\ \hline

\texttt{-last} \textit{frame}$|$\texttt{last}$|$\texttt{now}
& The last frame to use (default: \texttt{now}).
\\ \hline

\texttt{-all}[\texttt{frames}]
& Equivalent to \texttt{-first first -last last}.
\\ \hline

\texttt{-now}
& Equivalent to \texttt{-first now -last now}.
\\ \hline

\texttt{-namd}$|$\texttt{-vmd}
& Format of the unit cell parameters. When \texttt{-vmd}
is used, a parameter set will contains the VMD unitcell
parameters \textit{a}, \textit{b}, \textit{c}, \textit{alpha},
\textit{beta}, \textit{gamma}. When \texttt{-namd} is used, a
parameter set contains the three 3D unit cell vectors \textit{A},
\textit{B} and \textit{C} (default: \texttt{-vmd}).
\\ \hline  

\texttt{-}[\texttt{no}]\texttt{check}
& Check whether the unit cell parameters seem reasonable, \ie whether
the side lengths are not too small and the angles are not very small
or very large (default: \texttt{-nocheck}).
\\ \hline  

\end{tabular}

\newpage
\section{\texttt{readxst} and \texttt{writexst} -- Handling XST files}
\label{sec:writexst}
\label{sec:readxst}

\minisec{Syntax}
\texttt{pbc} \texttt{readxst} \textit{xstfile} [\textit{options}\dots]\\
\texttt{pbc} \texttt{writexst} \textit{xstfile} [\textit{options}\dots]

\minisec{Description}
Read/write the unit cell information from an XST or XSC file.

\minisec{Example}
\begin{Verbatim}
# read the unit cell parameters from system.xst
pbc readxst system.xst
pbc writexst system.xst
\end{Verbatim}

\minisec{Options}

\begin{tabular}{|p{0.3\textwidth}|p{0.64\textwidth}|}
\hline

\texttt{-molid} \textit{molid}$|$\texttt{top}
& Which molecule to use (default: \texttt{top}).
\\ \hline

\texttt{-first} \textit{frame}$|$\texttt{first}$|$\texttt{now}
& The first frame to use (default: \texttt{first}).
\\ \hline

\texttt{-last} \textit{frame}$|$\texttt{last}$|$\texttt{now}
& The last frame to use (default: \texttt{last}).
\\ \hline

\texttt{-all}[\texttt{frames}]
& Equivalent to \texttt{-first first -last last}.
\\ \hline

\texttt{-now}
& Equivalent to \texttt{-first now -last now}.
\\ \hline

\texttt{-stride} \textit{stride}
& Read only every \textit{stride}-th timestep from the
file (default: 1).
\\ \hline

\texttt{-}[\texttt{no}]\texttt{skipfirst}
& \emph{(only readxst)} Whether to skip the first line of the file, or not
(default: \texttt{-skipfirst} for XST files, \texttt{-noskipfirst} for
XSC files)
\\ \hline

\texttt{-step2frame} \textit{num}
& Conversion factor between step \textit{num} in XST file
and frame \textit{num} in DCDs. This is useful when loading multiple
XSTs and want to avoid over-writing info of earlier frames
by having a unique mapping between step and frame.
\\ \hline

\texttt{-step0} \textit{num}
& \emph{(only writexst)} Timestep number for the first written frame.
\\ \hline

\texttt{-}[\texttt{no}]\texttt{alignx}
& \emph{(only readxst)} If the unit cell vector \textit{A} is not
parallel to the x-axis, \texttt{-alignx} will rotate the system so that it is. If
\texttt{-noalignx} is used, the function will return with a warning
when \textit{A} ist not aligned with the x-axis.
\\ \hline

\texttt{-log} \textit{logfile}
& \emph{(only readxst)} Log file used for debugging information.
\\ \hline
\end{tabular}

\newpage
\section{\texttt{wrap} -- Wrapping atoms}
\label{sec:wrap}

\minisec{Syntax}

\texttt{pbc} \texttt{wrap} [\textit{options}\dots]

\minisec{Description} 

Wrap atoms into a single unitcell.

\minisec{Example}

\begin{Verbatim}
# wrap the system into the orthorhombic box shifted by one box length in X-dir
pbc wrap -orthorhombic -shiftcenterrel {1 0 0}
\end{Verbatim}


\minisec{Options}

\begin{longtable}{|p{0.3\textwidth}|p{0.64\textwidth}|}

\multicolumn{2}{r}{\dots}
\endfoot
\endlastfoot

\hline

\texttt{-molid} \textit{molid}$|$\texttt{top}
& Which molecule to use (default: \texttt{top})
\\ \hline

\texttt{-first} \textit{frame}$|$\texttt{first}$|$\texttt{now}
& The first frame to use (default: \texttt{now}).
\\ \hline

\texttt{-last} \textit{frame}$|$\texttt{last}$|$\texttt{now}
& The last frame to use (default: \texttt{now}).
\\ \hline

\texttt{-all}[\texttt{frames}]
& Equivalent to \texttt{-first first -last last}.
\\ \hline

\texttt{-now}
& Equivalent to \texttt{-first now -last now}.
\\ \hline

\texttt{-parallelepiped} $|$\texttt{-orthorhombic}
& Wrap the atoms into the unitcell parallelepiped or the corresponding
orthorhombic box with the same volume and center as the
(non-orthrhombic) unitcell. The unitcell displacement vectors are not
changed (default: \texttt{-parallelepiped}).
\\ \hline

\texttt{-sel} \textit{sel}
& The selection of atoms to be wrapped (default: \texttt{"all"}). Use
this if you don't want to wrap all atoms.
\\ \hline

\texttt{-nocompound}\linebreak $|$\texttt{-compound} \texttt{res}[\texttt{id}[\texttt{ue}]]$|$\texttt{seg}[\texttt{id}]$|$\texttt{chain}$|$\texttt{fragment}
& Defines, which atom compounds should be kept together, \ie which
atoms will not be wrapped if a compound would be split by the
wrapping: residues, segments or chains (default:
\texttt{-nocompound}).
\\ \hline

\texttt{-nocompoundref} $|$\texttt{-compoundref} \textit{refsel}
& When compounds have been defined via the \texttt{-compound} option,
this defines a reference selection of atoms. After the wrapping, at
least one of the atoms in this selection will be in the central
image. This can be useful, for example, when water molecules should be
wrapped such that the oxygen atom ends up in the central image
(default: \texttt{-nocompoundref}).
\\ \hline

\texttt{-center} \texttt{origin}$|$\texttt{unitcell}\linebreak
\hspace*{1em}$|$\texttt{com}$|$\texttt{centerofmass}\linebreak 
\hspace*{1em}$|$\texttt{bb}$|$\texttt{boundingbox}
& Specify the center of the wrapping cell. The center can be set to
the origin (\texttt{origin}), to the center of the unit cell
(\texttt{unitcell}), to the center of mass (\texttt{com} or
\texttt{centerofmass}) of the selection specified by the option
\texttt{-centersel}, or to the center of the bounding box (\texttt{bb}
or \texttt{boundingbox}) of the selection specified by the option
\texttt{-centersel} (default: \texttt{unitcell}).
\\ \hline

\texttt{-centersel} \textit{sel}
& Specify the selection \textit{sel} that defines the center of the
wrapping cell in the option \texttt{-center} (default:
\texttt{"all"}). Note that this option only has an effect if used
together with the arguments \texttt{com}, \texttt{centerofmass},
\texttt{bb} or \texttt{boundingbox} to the option \texttt{-center}.
\\ \hline

\texttt{-shiftcenter} \textit{shift}
& Shift the center of the box by \textit{shift}. \textit{shift} has to
be a list of three numerical values. (default: \verb!{0 0 0}!)
\\ \hline

\texttt{-shiftcenterrel} \textit{shift}
& Shift the center of the box by \textit{shift} (in units of
the unit cell vectors). \textit{shift} has to be a list of
three numerical values. (default: \verb!{0 0 0}!)
\\ \hline

\texttt{-}[\texttt{no}]\texttt{verbose}
& Turn on/off verbosity of the function (for debugging) (default:
\texttt{-noverbose}).
\\ \hline

\texttt{-}[\texttt{no}]\texttt{draw}
& Draw some test vectors (for debugging) (default: \texttt{-nodraw}).
\\ \hline

\end{longtable}

\newpage
\section{\texttt{unwrap} -- Unwrapping atoms}
\label{sec:unwrap}
\minisec{Syntax}

\texttt{pbc} \texttt{unwrap} [\textit{options}\dots]

\minisec{Description} 

If a simulation only saves the central image coordinates of a system,
atoms are wrapped around when they reach the boundaries.  This leads
to big jumps in the coordinates of the atoms, and to bonds that
stretch the whole box length. This procedure will reverse these jumps
and make the movement of the atoms continuous over a series of
frames. This process is not necessarily unique, so this procedure can
\emph{not} exactly reverse the effects of the command \texttt{pbc
  wrap}.

In the case of a simulation trajectory, the following process most
probably gives the best result:
\begin{enumerate}
\item Go to the first frame.
\item \label{shape} Shape the unitcell of the frame for the best
  visualization by using the commands\\ \mbox{\texttt{pbc join -now}}
  and \mbox{\texttt{pbc wrap -now}} with appropriate options.
\item Unwrap the trajectory, starting from the current frame, by using\\
  \mbox{\texttt{pbc unwrap -first now}}.
\item Visually check the result. If the system gets smeared out too
  fast because the diffusion is too high, repeat the procedure with
  successively later frames.
\end{enumerate}

\minisec{Example}

\begin{Verbatim}
# unwrap all protein atoms
pbc unwrap -sel "protein"
\end{Verbatim}

\minisec{Options}

\begin{tabular}{|p{0.3\textwidth}|p{0.64\textwidth}|}
\hline

\texttt{-molid} \textit{molid}$|$\texttt{top}
& Which molecule to use (default: \texttt{top})
\\ \hline

\texttt{-first} \textit{frame}$|$\texttt{first}$|$\texttt{now}
& The first frame to use (default: \texttt{now}).
\\ \hline

\texttt{-last} \textit{frame}$|$\texttt{last}$|$\texttt{now}
& The last frame to use (default: \texttt{now}).
\\ \hline

\texttt{-all}[\texttt{frames}]
& Equivalent to \texttt{-first first -last last}.
\\ \hline

\texttt{-now}
& Equivalent to \texttt{-first now -last now}.
\\ \hline

\texttt{-sel} \textit{sel}
& The selection of atoms to be unwrapped (default: \texttt{"all"}). Use
this if you don't want to unwrap all atoms.
\\ \hline

\texttt{-}[\texttt{no}]\texttt{verbose}
& Turn on/off verbosity of the function (for debugging) (default:
\texttt{-noverbose}).
\\ \hline
\end{tabular}


\newpage
\section{\texttt{join} -- Joining residues, chains, segments, fragments,
  and connected/bonded groups}
\label{sec:join}

\minisec{Syntax}

\texttt{pbc} \texttt{join} \textit{compound} [\textit{options}\dots]

\minisec{Description} 

Joins compounds of type \textit{compound} of atoms that have been
split due to wrapping around the unit cell boundaries, so that they
are not split anymore. \textit{compound} must be one of the values
\texttt{res[id[ue]]}, \texttt{chain}, \texttt{seg[id]},
\texttt{fragment} or \texttt{connected}.

This procedure can help to remove bonds that stretch the whole box.
Note, however, that \texttt{join} is relatively slow and is required
only in few cases.  If you have a simulation trajectory that contains
frames with overstretched bonds, it is usually enough to apply
\texttt{join} only to the first frame and then the much faster
procedure \texttt{unwrap} to all of the frames:
\begin{Verbatim}
  pbc join \textit{compound} -first 0 -last 0 
  pbc unwrap
\end{Verbatim}

Note also, that the (faster) default algorithm only works when none of
the compounds stretches more than half the periodic box in any direction.
With the option \texttt{-bondlist} you can select an alternate algorithm
that joins compounds based on direct bonds and does not suffer from this
limitation, but can be significantly slower.

\minisec{Examples}

\begin{Verbatim}
# join all residues such that the Carbon alpha atom 
# is in the central image
pbc join res -ref "name CA"
# join all bonds of long polymer chain molecules
pbc join fragment -bondlist
\end{Verbatim}

\minisec{Options}

\begin{longtable}{|p{0.3\textwidth}|p{0.64\textwidth}|}
\multicolumn{2}{r}{\dots}
\endfoot
\endlastfoot

\hline

\texttt{-molid} \textit{molid}$|$\texttt{top}
& Which molecule to use (default: \texttt{top})
\\ \hline

\texttt{-first} \textit{frame}$|$\texttt{first}$|$\texttt{now}
& The first frame to use (default: \texttt{now}). 
\\ \hline

\texttt{-last} \textit{frame}$|$\texttt{last}$|$\texttt{now}
& The last frame to use (default: \texttt{now}).
\\ \hline

\texttt{-all}[\texttt{frames}]
& Equivalent to \texttt{-first first -last last}.
\\ \hline

\texttt{-now}
& Equivalent to \texttt{-first now -last now}.
\\ \hline

\texttt{-sel} \textit{sel}
& The selection of atoms to be joined (default: \texttt{"all"}). Use
this if you don't want to join all atoms.
\\ \hline

\texttt{-noborder}$|$\texttt{-border} \textit{depth}
& When only atoms close to the boundaries of the unit cell have
overstretched bonds, this option can be used to specify the maximal
depth inside the system. Using this option will significantly speed up
join (default: \texttt{-noborder}).
\\ \hline

\texttt{-noref}$|$\texttt{-ref} \textit{refsel}
& This defines a reference selection of atoms. When joining compounds,
the first atom matching the selection in each compound will be chosen,
and all atoms will be wrapped into a unit cell around this atom. If
\texttt{noref} is used, the first atom in the compound is the
reference atom (default: \texttt{-noref}).
\\ \hline

\texttt{-}[\texttt{no}]\texttt{bondlist}
& Turn on/off alternate, bond topology based joining algorithm (default:
\texttt{-nobondlist}).
\\ \hline

\texttt{-}[\texttt{no}]\texttt{verbose}
& Turn on/off verbosity of the function (for debugging) (default:
\texttt{-noverbose}).
\\ \hline
\end{longtable}

\newpage
\section{\texttt{box} and \texttt{box\_draw} -- Drawing the unit cell boundaries}

\subsection{\texttt{box} -- Automatically updateing box}
\label{sec:box}

\minisec{Syntax}

\texttt{pbc} \texttt{box} [\textit{options}\dots]

\minisec{Description} 

(Re)Draws a box that shows the boundaries of the unit cell. The box
will automatically adapt to changes in the unit cell parameters in the
course of a trajectory, as for example for simulations at constant
pressure. Only a single automatically updated box can exist at a time.

\minisec{Example}

\begin{Verbatim}
# draw a box, centered on the origin
pbc box -center origin
\end{Verbatim}

\minisec{Options}

\begin{tabular}{|p{0.3\textwidth}|p{0.64\textwidth}|}
\hline

\texttt{-molid} \textit{molid}$|$\texttt{top}
& Which molecule to use (default: \texttt{top})
\\ \hline

\texttt{-on}$|$\texttt{-off}$|$\texttt{-toggle}
& Turn the box on, off, or toggle whether it is on or off. (default:
\texttt{-on})
\\ \hline

\texttt{-parallelepiped} $|$\texttt{-orthorhombic}
& Draw the box as a parallelpiped, or as the corresponding
orthorhombic box. (default: \texttt{-parallelepiped}).
\\ \hline

\texttt{-color} \textit{color}
& Draw the box in color \textit{color}. (default: \texttt{blue})
\\ \hline

\texttt{-material} \textit{Material}
& Draw the box using the material \textit{Material}. (default: \texttt{Opaque})
\\ \hline


\texttt{-style}
\texttt{lines}$|$\texttt{dashed}$|$\texttt{arrows}$|$\texttt{tubes}
& Choose the style of the box (default: \texttt{lines}).
\\ \hline

\texttt{-width} \textit{width}
& Define the \textit{width} of the lines/arrows/tubes (default:
\texttt{3}). 
\\ \hline

\texttt{-resolution} \textit{res}
& Use \textit{resolution} faces for the tube style (default:
\texttt{8}).
\\ \hline

\texttt{-center} \texttt{origin}$|$\texttt{unitcell}\linebreak
\hspace*{1em}$|$\texttt{com}$|$\texttt{centerofmass}\linebreak 
\hspace*{1em}$|$\texttt{bb}$|$\texttt{boundingbox}
& Specify the center of the box. The center can be set to
the origin (\texttt{origin}), to the center of the unit cell
(\texttt{unitcell}), to the center of mass (\texttt{com} or
\texttt{centerofmass}) of the selection specified by the option
\texttt{-centersel}, or to the center of the bounding box (\texttt{bb}
or \texttt{boundingbox}) of the selection specified by the option
\texttt{-centersel} (default: \texttt{unitcell}).
\\ \hline

\texttt{-centersel} \textit{sel}
& Specify the selection \textit{sel} that defines the center of the
wrapping cell in the option \texttt{-center} (default:
\texttt{"all"}).
\\ \hline

\texttt{-shiftcenter} \textit{shift}
& Shift the center of the box by \textit{shift}. \textit{shift} has to
be a list of three numerical values. (default: \verb!{0 0 0}!)
\\ \hline

\texttt{-shiftcenterrel} \textit{shift}
& Shift the center of the box by \textit{shift} (in units of
the unit cell vectors). \textit{shift} has to be a list of
three umerical values. (default: \verb!{0 0 0}!)
\\ \hline
\end{tabular}

\newpage
\subsection{\texttt{box\_draw} -- Drawing a static box}
\label{sec:box_draw}

\minisec{Syntax}

\texttt{pbc} \texttt{box\_draw} [\textit{options}\dots]

\minisec{Description} 

Draws a static box that shows the boundaries of the unit cell, but
will not adapt to changes in the unitcell properties. This might be
useful when you want to draw more than one box at a time (\eg to show
periodic images of a box), or to show the initial box in a simulation
with fluctuating box unit cell geometry.

\minisec{Options} 

\texttt{pbc box\_draw} uses the same options as the command
\texttt{pbc box}, with the exception of the options
\texttt{-on}$|$\texttt{-off}$|$\texttt{-toggle} and \texttt{-color},
which can not be used. To set the color of the box, use the
\texttt{graphics color} command.

\minisec{Example}

\begin{Verbatim}
# draw a box around the central image 
set box0 [pbc box_draw -shiftcenterrel { 0 0 0 }] 
# draw a box around the central image shifted by 
# the unit cell vector C 
set box1 [pbc box_draw -shiftcenterrel { 0 0 1 }]
\end{Verbatim}

\section{Credits}

The \pbctools plugin has been written by (in alphabetical order)
\begin{itemize}
\item Toni Giorgino \texttt{<toni.giorgino \_at\_ isib.cnr.it>}
\item Jerome Henin \texttt{<jhenin \_at\_ cmm.upenn.edu>}
\item Olaf Lenz \texttt{<olenz \_at\_ icp.uni-stuttgart.de>} (maintainer)
\item Cameron Mura \texttt{<cmura \_at\_ mccammon.ucsd.edu>}
\item Jan Saam \texttt{<saam \_at\_ charite.de>}
\end{itemize}
The \texttt{pbcbox} procedure copies a lot of the ideas of Axel
Kohlmeyer's script \texttt{vmd\_draw\_unitcell}.
Please submit your bug reports, comments and feature requests on the
\pbctools
homepage\footnote{\url{http://github.com/olenz/pbctools/}}.

\end{document}
